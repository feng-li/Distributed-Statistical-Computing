\section{Spark实例:决策树模型}\label{ux5b9eux4f8bux5206ux6790ux672fux540eux75c5ux4ebaux5b89ux7f6eux65b9ux6848ux7684ux51b3ux7b56ux6811ux6a21ux578b}

\subsection{准备知识}\label{ux51c6ux5907ux77e5ux8bc6}

\begin{itemize}
\itemsep1pt\parskip0pt\parsep0pt
\item
  决策树模型
\item
  Python和Spark
\end{itemize}

\subsection{数据介绍}\label{ux6570ux636eux4ecbux7ecd}

\subsubsection{数据来源}\label{ux6570ux636eux6765ux6e90}

本文研究数据是由 UC Irvine Machine Learning Repository,即加州大学欧文分校机器学习库提供
的 1993年某医院关于病人手术后体征表现的数据
集\texttt{(http://archive.ics.uci.edu/ml/datasets/Post-Operative+Patient)}。该数据集的主要
分类任务是决定术后恢复区的病人是否应当被转移到特护病房、普通病房,或者直接出院回家。由于体温
降低是衡量术后恢复状况的一个重要指标,因此本数据集中的病人体征指标主要考察病人手术后的体温水
平。

\subsubsection{数据预处理}\label{ux6570ux636eux9884ux5904ux7406}

原始数据集包含 90 个观测,9 个变量,其中有 1 个因变量,8个自变量。由于原始数据集中因变量一
列 中包含 3 个缺失数据,因此删去这 3个观测,最终得到含 87 个观测、9 个变量的有效数据。为了方
便后续建模过程的运算,我们依次将分类变量的字符串值分别转化成数值型,如将“低”、“中”、“高”分别
记为“1”、“2”、“3”。

\subsection{建立模型}\label{ux5efaux7acbux6a21ux578b}

\subsubsection{数据读入}\label{ux6570ux636eux8bfbux5165}

由于 Spark 中用于建立决策树的数据集必须为 LabeledPoint格式,因此我们在读入 csv 格式的数据之
后,必须对现有数据集进行格式转换。LabeledPoint 是 Spark中独有的一种标签数据集格式,数据结 构
分为 label 和features两部分。具体结构为\texttt{label index1:value1 index2:value2
  \ldots{}},其中 label为标签数据,index1,index2 为特征值序号,value1,value2 为特征值。

在 Pyspark 中,我们可以用\texttt{pyspark.mllib.regression}模块中的“LabeledPoint”函数对读入的
数据集中的每一行进行格式转换,生成模型可识别的数据格式。具体代码如下:

\begin{lstlisting}
from __future__ import print_function
from pyspark import SparkContext
from pyspark.mllib.tree import DecisionTree, DecisionTreeModel
from pyspark.mllib.regression import LabeledPoint
####定义格式转化函数 datatrans###
def datatrans(d):
    t = [m for m in d.split(',')]
    y = float(t[0])-1
    x = [float(t) for t in t[1:]]
    return LabeledPoint(y,x)
####对读入的数据集 mydata 的每一行进行格式转化###
mydata = sc.textFile("/home/dmc/finalHWdata.csv")
data = mydata.map(datatrans)
\end{lstlisting}

\subsubsection{建立决策树}\label{ux5efaux7acbux51b3ux7b56ux6811}

将已转化成 LabeledPoint 格式的数据集 data 分成 3:1
的两份,一份为训练集(75\%),一份为测试集
(25\%)。利用训练集的数据进行建模,选用 Gini
不纯度作为节点纯度的度量。具体代码如下:

\begin{lstlisting}
(trainingData, testData) = data.randomSplit([0.75, 0.25]) ###生成训练集和测试集
Model = DecisionTree.trainClassifier(trainingData, numClasses=3,\
categoricalFeaturesInfo={}, impurity='gini', maxDepth=5, maxBins=32)
\end{lstlisting}

\subsubsection{在测试集上预测}\label{ux5728ux6d4bux8bd5ux96c6ux4e0aux9884ux6d4b}

基于训练集中建立好的决策树模型,用\texttt{predict}
函数对预测集进行预测。得到预测集上的错分率Test Error为
39.29\%,具体代码如下:

\begin{lstlisting}
predictions = model.predict(testData.map(lambda x: x.features))
######计算错分率######
labelsAndPredictions = testData.map(lambda lp: lp.label).zip(predictions)
testErr = labelsAndPredictions.filter(lambda (v, p): v != p).coun\
          t()/ float(testData.count())
print('Test Error = ' + str(testErr))
print('Learned classification tree model: ' + model.toDebugString())
\end{lstlisting}
