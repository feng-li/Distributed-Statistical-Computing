\section{Mahout实例:Logistics回归和随机森林分类算法}\label{ux5b9eux4f8bux5206ux6790mahout-ux673aux5668ux5b66ux4e60ux7814ux7a76ux7ea2ux9152ux8d28ux91cfux5f71ux54cdux56e0ux7d20}

\subsection{准备知识}\label{ux51c6ux5907ux77e5ux8bc6}

\begin{itemize}
\itemsep1pt\parskip0pt\parsep0pt
\item
  Logistics回归和随机森林分类算法
\item
  Mahout和R
\item
  Hadoop Streaming
\end{itemize}

\subsection{研究背景}\label{ux7814ux7a76ux80ccux666f}

本案例主要使用 Hadoop 和 Mahout 的机器学习算法对实际数据完成分类问题的学习。为了实践机器 学
习算法,使用 UCI机器学习库的红酒质量数据集,学习的内容为根据红酒的一些属性判断红酒质量水平。
在这里使用的分类方法为 Logistic 回归、随机森林。从 Logistic回归得到的结果发现酒精浓 度、密
度、糖分量等和红酒质量是正相关的,氯化钠含量、柠檬酸含量、挥发性酸含量等和红酒质量是 负相关
的。通过随机森林的学习,我们依据红酒属性得到对红酒质量的判断可以达到很高的准确性。

\subsection{数据准备}\label{ux6570ux636eux51c6ux5907}

\subsubsection{数据信息}\label{ux6570ux636eux4fe1ux606f}

本次报告分析的数据集来自UCI机器学习数据库的红酒质量数据集(网
址:\path{http://archive.ics.uci.edu/ml/machine-learning-databases/wine-quality/})。该数据
集包含 1599 个样本,13 个变量,其中 1 个变量为样本序号、11个变量为红酒的一些属性数值描
述、1 个变量为该红酒样本的质量评分。

\subsubsection{数据预处理}\label{ux6570ux636eux9884ux5904ux7406}

从 UCI 机器学习库直接下载得到的数据集 winequality-red.csv
是分号分割格式的数据,且没有变量名信息,参考它提供的文件
winequality.names,使用 R 读入数据,并添加变量名信息。为了比较不同
机器学习的算法,首先,使用分层随机抽样的方式(根据因变量quality
分层),取数据集的 70\% 为训练集,剩 下的 30\% 作为测试集。拆分后,训练集
共有 1120 个样本,测试集共有 479 个样本,分别保存为文件 train\_redwine.csv
和 test\_redwine.csv。由于Logistic回归要求的因变量为二分类变量,而这里因变量
quality可以有3、4、5、 6、7、8 共 6 种取值,分别代表红酒的质量从低到高。
这里粗略地将响应变量 quality 分为两类,其中品质评分 3、4、5
的归为一类,用 0 表示,表示红酒品质一般;品质评分 6、 7、8 的归为一类,用 1
表示,表示红酒品质较好。

\subsection{使用 Mahout 计算 Logistic
回归}\label{ux4f7fux7528-mahout-ux8ba1ux7b97-logistic-ux56deux5f52}

Logistic 回归常是一种有效的分类问题学习方法,并且模型可解释性较好。
然而它的求解算法是迭代的,在 Hadoop 分布式计算中的 MapReduce
一般要求算法是可以拆分并行 进行的,也就是要求算法不是迭代的。
所以要分布式计算 Logistic 回归的解,简单的对 Logistic
回归求解的迭代算法拆分成 Map 和 Reduce 两个部分是不合理的。所以要在
Hadoop 上使用 MapReduce 计算 Logistic 回归的解,一般定义 Mapper 函数 和
Reducer 函数为:

\begin{enumerate}
\def\labelenumi{\arabic{enumi}.}
\item
  Mapper:将n个样本的数据集拆分成k个子集,分别在k个子集上使用随机梯度下降(Stochastic
  Gradient Descent,简称 SGD)算法计算 Logistic 回归的解。
\item
  Reducer:计算Mapper中得到的k个Logistic回归解的均值作为Logistic回归的解。
\end{enumerate}

在数学上可以证明,当样本数 n 趋于无穷时,以上得到的 Logistic
回归解有相合性,且该估计量的分布是渐进正态的。并且实际计算中,单个程序中
Logistic 回归解的求解使用的SGD算法的计算速度是很快的。
所以以上提出的MapReduce算法计算Logistic解可以
做到维持估计准确性的同时提高计算速度。

\subsubsection{Logistic
回归模型训练}\label{logistic-ux56deux5f52ux6a21ux578bux8badux7ec3}

因为在这个问题中我们使用的数据集样本量仅 1000
多,所以为了计算的准确性,仅使用单机模式计算模 型的解。在 Linux 系统下的
Mahout 下计算 Logistic 的解的代码如下。

\begin{lstlisting}
	$ mahout trainlogistic --input train_redwine.csv \
	--output ./logit_model \
	--target quality --categories 2 \
	--predictors fixed.acidity volatile.acidity citric.acid \
	residual.sugar chlorides free.sulfur.dioxide total.sulfur.dioxide \
	density pH sulphates alcohol --types numeric \
	--features 20 --passes 100
\end{lstlisting}

以上代码在存放数据集train\_redwine.csv和test\_redwine.csv的目录下运行。
其中使用的选项包括:

\begin{itemize}
\item
  \lstinline!trainlogistic!代表计算Logistic回归的模型解
\item
  \lstinline!--input!选择输入的数据集为train\_redwine.csv
\item
  \lstinline!--output!代表将结果输出到当前目录下的文件logit\_model中
\item
  \lstinline!--target!说明
  因变量为quality,并告知Mahout系统它的类型为2分类的分类变量
\item
  \lstinline!--predictors!说明自变量为紧接
  着的11个变量,并告知Mahout它们都是数值型变量
\item
  \lstinline!--features!说明建模型时使用的的内部特征向量 大小
\item
  -\lstinline!-passes!说明在学习模型的时候输入数据需要重检查的次数
\end{itemize}

输出结显示对红酒品质有正影响的变量包括酒精浓度、柠檬酸
含量、密度、非挥发性酸含量、游离二
氧化硫含量、酸碱度、残余糖分量、硫酸钾含量;
对红酒品质有负影响的变量包括氯化钠含量、柠檬酸
含量、总二氧化硫含量、挥发性酸含量。事实上,Logistic
回归得到的结果对各属性对品质影响的解释 与显示是相符的。

\subsubsection{Logistic
回归模型测试}\label{logistic-ux56deux5f52ux6a21ux578bux6d4bux8bd5}

通过以上得到的 Logistic 模型对测试集做模型检测的代码如下。

\begin{lstlisting}
	$ mahout runlogistic --input test_redwine.csv \
	--model ./logit_model --auc --confusion
\end{lstlisting}

以上代码中使用的选项包括:

\begin{itemize}
\item
  \lstinline!runlogistic!代表通过测试集检验之前得到的Logistic回归的模型解的效果
\item
  \lstinline!--input!说明使用的测试集为test\_redwine.csv
\item
  \lstinline!--model!说明使用的Logistic回归模型结果文件为logit\_model
\item
  \lstinline!--auc!表示输出模型的AUC得分,即ROC曲线下方的面积占比
\item
  \lstinline!--confusion!表示输出混淆矩阵
\end{itemize}

输出结果显示该模型的 AUC 得分为 0.64 \textgreater{} 0.5
,说明训练的效果还可以。混淆矩阵显示品质为 0 的 223 个红酒样本有 148
个被判断为 0,75 个判断为 1;品质 为 1 的 256 个红酒样本有 121 个被判断 为
0,135 个被判断为 1。从混淆矩阵显示的误判 率可以看出 Logistic
回归的模型学习结果不够令人 满意。

\subsection{使用 Mahout
计算随机森林}\label{ux4f7fux7528-mahout-ux8ba1ux7b97ux968fux673aux68eeux6797}

随机森林是一种机器学习集成算法,它的思想基于使用多个弱分类器得到强分类器。
随机森林算法基于
决策树的算法。随机森林算法事实上建立的多个决策树是可以并行计算的。在测试结果的时候,也可以
并行计算多个个决策树的分类结果。所以要在 Hadoop 上使用 MapReduce
建立随机森林的模型,一般定 义 Mapper 函数和 Reducer 函数为:1.
Mapper:在各个子节点上并行地分别抽取训练样本和输入变量,建立决策树;2.
Reducer:整合各个子节点计算得到的多个决策树。

\subsubsection{数据准备}\label{ux6570ux636eux51c6ux5907-1}

在 Linux 系统下的 Mahout
下并行计算随机森林模型要求的输入数据集是纯数据集,
不包含变量信息,所以我们将 Logistic 回归中使用的数据集 train\_redwine.csv
和 test\_redwine.csv
删除首行变量名信息,并删除红酒样本序号一列,得到数据集 redwine\_train.arff
和 redwine\_test.arff。为了使用 Hadoop 并行计算完成 Mahout
的随机森林的计算,首先要将数据集 redwine\_train.arff 和
redwine\_test.arff 放入 HDFS 中,代码如下:

\begin{lstlisting}
	$ hadoop fs –put redwine*
\end{lstlisting}

因为数据集 redwine\_train.arff 和 redwine\_test.arff
本身没有变量信息,要在随机森林 算法中学 习,需要建立变量信息文件,可以使用
Mahout 中的 Describe 工具创建变量信息 文件,代码如下:

\begin{lstlisting}
	$ hadoop jar $MAHOUT_HOME/mahout-examples-0.10.1-job.jar \
	org.apache.mahout.classifier.df.tools.Describe \
	-p redwine_train.arff \
	-f redwine_train.info \
	-d 11 N L
\end{lstlisting}

以上代码使用 Hadoop 运行 Mahout 的方法
classifier.df.tools.Describe,其中使用的选项包括:

\begin{itemize}
\item
  \lstinline!-p!  说明要生成的变量信息用于描述数据集redwine\_train.arff
\item
  \lstinline!-f! 说明生成的变量信息保存为文件redwine\_train.info
\item
  \lstinline!-d! 说明数据集中的变量信息,后面的内容意思为该数据集中的变量前11个都是数值
  型,最后一个是 因变量可以通过如下语句查看生成的变量信息文件
  redwine\_train.info 内容:

\begin{lstlisting}
$ hadoop fs -cat redwine_train.info
\end{lstlisting}
\end{itemize}

为了训练集合变量描述两个数据集在并行计算中其他计算机的使用,需要修改它们的使用权限,代码如下:

\begin{lstlisting}
	$ hadoop fs -chmod 751 redwine_train.arff
	$ hadoop fs -chmod 751 redwine_train.info
\end{lstlisting}

\subsubsection{随机森林模型训练}\label{ux968fux673aux68eeux6797ux6a21ux578bux8badux7ec3}

在 Hadoop 中并行计算 Mahout 的随机森林模型代码如下:

\begin{lstlisting}
	$ hadoop jar $MAHOUT_HOME/mahout-examples-0.10.1-job.jar \
	org.apache.mahout.classifier.df.mapreduce.BuildForest \
	-d redwine_train.arff \
	-ds redwine_train.info \
	-o redwine_forest \
	-sl 5 -t 100
\end{lstlisting}

以上代码使用 Hadoop 的 MapReduce 运行 Mahout
的机器学习算法建立随机森林,其中使 用的选项包括:

\begin{itemize}
\item
  \lstinline!-d! 说明使用的训练集redwine\_train.arff
\item
  \lstinline!-ds! 说明使用的变量描述文件redwine\_train.info
\item
  \lstinline!-o!说明将计算得到的随机森林模型结果输出到目录redwine\_forest下
\item
  \lstinline!-sl! 说明建立每个决策树时随机选择的自变量数,在此定义为5
\item
  \lstinline!-t! 说明建立的决策树数目,在此定义为100
\end{itemize}

\subsubsection{随机森林模型测试}\label{ux968fux673aux68eeux6797ux6a21ux578bux6d4bux8bd5}

使用测试集测试如上生成的随机森林模型的效果代码如下:

\begin{lstlisting}
	$ hadoop jar $MAHOUT_HOME/mahout-examples-0.10.1-job.jar \
	org.apache.mahout.classifier.df.mapreduce.TestForest \
	-i redwine_test.arff \
	-ds redwine_train.info \
	-m redwine_forest \
	-o redwine_prediction \
	-a
\end{lstlisting}

其中使用的选项包括

\begin{itemize}
\item
  \lstinline!-i! 说明使用的测试集redwine\_test.arff
\item
  \lstinline!-ds! 说明使用的变量描述文件redwine\_train.info
\item
  \lstinline!-m! 说明测试的模型来自目录redwine\_forest
\item
  \lstinline!-o! 说明测试输出的结果输出到目录redwine\_prediction下
\item
  \lstinline!-a! 要求输出混淆矩阵
\end{itemize}

得到的随机森林模型测试输出结果可以看到随机森林的正确率高达 98.33\%。Logistic 模型的测试结
果 的混淆矩阵,随机森林模型的准确性十分高。虽然随机森林的判断结果准确性更高,但是模型解释性
相 较 Logistic回归模型较差,我们没有办法通过输出的结果说明红酒的各个属性变量在判断红酒质量
方 面的贡献。
